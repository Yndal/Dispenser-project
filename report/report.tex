\documentclass{sigchi}

% Use this command to override the default ACM copyright statement (e.g. for preprints). 
% Consult the conference website for the camera-ready copyright statement.


%% EXAMPLE BEGIN -- HOW TO OVERRIDE THE DEFAULT COPYRIGHT STRIP -- (July 22, 2013 - Paul Baumann)
% \toappear{Permission to make digital or hard copies of all or part of this work for personal or classroom use is 	granted without fee provided that copies are not made or distributed for profit or commercial advantage and that copies bear this notice and the full citation on the first page. Copyrights for components of this work owned by others than ACM must be honored. Abstracting with credit is permitted. To copy otherwise, or republish, to post on servers or to redistribute to lists, requires prior specific permission and/or a fee. Request permissions from permissions@acm.org. \\
% {\emph{CHI'14}}, April 26--May 1, 2014, Toronto, Canada. \\
% Copyright \copyright~2014 ACM ISBN/14/04...\$15.00. \\
% DOI string from ACM form confirmation}
%% EXAMPLE END -- HOW TO OVERRIDE THE DEFAULT COPYRIGHT STRIP -- (July 22, 2013 - Paul Baumann)


% Arabic page numbers for submission. 
% Remove this line to eliminate page numbers for the camera ready copy
% \pagenumbering{arabic}


% Load basic packages
\usepackage{balance}  % to better equalize the last page
\usepackage{graphics} % for EPS, load graphicx instead
\usepackage{times}    % comment if you want LaTeX's default font
\usepackage{url}      % llt: nicely formatted URLs
\usepackage[utf8]{inputenc}
\usepackage{color}
%\usepackage{tabulary}
%\usepackage{amsmath}
\usepackage{array}


% llt: Define a global style for URLs, rather that the default one
\makeatletter
\def\url@leostyle{%
	\@ifundefined{selectfont}{\def\UrlFont{\sf}}{\def\UrlFont{\small\bf\ttfamily}}}
\makeatother
\urlstyle{leo}


% To make various LaTeX processors do the right thing with page size.
\def\pprw{8.5in}
\def\pprh{11in}
\special{papersize=\pprw,\pprh}
\setlength{\paperwidth}{\pprw}
\setlength{\paperheight}{\pprh}
\setlength{\pdfpagewidth}{\pprw}
\setlength{\pdfpageheight}{\pprh}

% Make sure hyperref comes last of your loaded packages, 
% to give it a fighting chance of not being over-written, 
% since its job is to redefine many LaTeX commands.
\usepackage[pdftex]{hyperref}
\hypersetup{
	pdftitle={SIGCHI Conference Proceedings Format},
	pdfauthor={LaTeX},
	pdfkeywords={SIGCHI, proceedings, archival format},
	bookmarksnumbered,
	pdfstartview={FitH},
	colorlinks,
	citecolor=black,
	filecolor=black,
	linkcolor=black,
	urlcolor=black,
	breaklinks=true,
}

% create a shortcut to typeset table headings
\newcommand\tabhead[1]{\small\textbf{#1}}

\definecolor{Orange}{rgb}{1,0.5,0}
\newcommand{\todo}[1]{\textsf{\textbf{\textcolor{Orange}{[[#1]]}}}}

\toappear{Paper submitted for evaluation in an elective project, spring 2016. The IT University of Copenhagen. Copyright remains with the author.}

% End of preamble. Here it comes the document.
\begin{document}
	
	\title{Towards the modular liquid-handling platform}
	
	\numberofauthors{1}
	\author{
		\alignauthor Lars Yndal Sørensen\\
		\affaddr{MSc. Stud., IT University of Copenhagen}\\
		\affaddr{Rued Langgaards Vej 7, 2300 Cph S}\\
		\email{lynd@itu.dk}\\
		%\affaddr{Optional phone number}
	}
	
	\maketitle
	
	\begin{abstract}
		TBD
		
		diskutere dem ud 
		fra kriterier som performance, kompleksistet/pris, vedligehold, udvidbarhed,
		osv. Derefter vende rundt og kigge på hvad for en type applikation vi kigger på og ser derefter hvad for en type der så passer bedst.
		
		Hvis der ikke i litteraturen findes eksempler på alle mulige typer er det
		ikke et problem snarere er det godt fordi der kunne være et hul der
		var værd at efterforske.
		
		Questions:\\
			\todo{1) Correct term: Dish dispenser, plate dispenser, similar...?}\\
			\todo{2) General term: Dishes or plates? (Must include both Petri dishes and wells.)}\\
			\todo{3) Write "I propose, or we propose...")}\\
			\todo{4) laboratory platform vs liguid-handling machine}\\
			\todo{5) Use term "experiment container" in instriduction to have as open approach as possible?}\\
			\todo{Mention: Not high performance design (?)}
			
	\end{abstract}
	
	\keywords{
		Guides; instructions; author's kit; conference publications;
		keywords should be separated by a semi-colon. \newline
%		\textcolor{red}{Optional section to be included in your final version, 
%			but strongly encouraged.}
	}
	
\category{H.5.m.}{Information Interfaces and Presentation (e.g. HCI)}{Miscellaneous}
	
	See: \url{http://www.acm.org/about/class/1998/}
	for more information and the full list of ACM classifiers
	and descriptors. \newline
%	\textcolor{red}{Optional section to be included in your final version, 
%		but strongly encouraged. On the submission page only the classifiers’ 
%		letter-number combination will need to be entered.}
	
	\section{Introduction}
	A lot of public and private institutions are working with laboratory platform to perform experiments in Petri dishes or wells. These experiments are often related to human health or other biological research, while each series of these experiments can be performed in a single type of dish. This has entailed that the available laboratory platforms mainly focus on to serve \textit{one} type of dishes and that the platforms are designed for a single type of experiments. 
	
	In the design process, the knowledge of having to address just a certain type of experiments, makes it more rigid, what such a platform requires of equipment. Therefore each platform are very often designed in a non-modular way as each piece of equipment seeks to address as many functionalities as possible.
	
	Public and private institutions that use laboratory platforms may, for several reasons, not be interested in platforms that aim at specific experiments. Reasons that are related to financial circumstances, or because they just need to perform a limited serie of experiments, before turning to a completely different type of experiments. This rules out the use of available laboratory platforms, in which case, some institutions are creating their own platforms to address their specific needs.
	
	If the laboratory platforms where less rigid in their design, in terms of modularity, and they used more generic approach, they would be able to perform experiments in \textit{a group} of types and address more needs per platform. Of course the specialized platforms would still be needed as some experiments simply can be generalized, but starting with, i.e. a module for adding and removing Petri dishes and wells, would benefit the majority, as this is the most common event on a laboratory platform.
	
	This paper will from this point on, focus on the design of a module that can insert and remove experiment containers, describing the everything from the initial design goals, through  the mechanical and software design, to the evaluation containing an integrated experiment into an laboratory platform\footnote{The platform being integrated into during the evaluating experiments is currently still being designed, but is far enough the process that actual experiments are being conducted on this.}.
	
	
	
	
	
%	Institutions that work with laboratory platforms, are often working with their own custom designed ones or have chosen to acquire one from a private company.
%	
%	Many of the platforms that are available are highly specialized and can therefore not handle more than a single type of dishes or wells. For those that come with a dish insertion/removal functionality, the end manipulator for handling the dishes are deeply integrated to the rest of the platform. This forces to users to acquire the platform as \textit{one} single unit instead of a modular design, that would address the users needs with a higher granularity.
%	
%	
%	
%	
%	
%	For most laboratory platforms that include a \todo{Petri dish or well square} end manipulation, the end manipulator is deeply integrated into the rest of the platform. Some times the manipulator may serves multiple purposes\todo{reference}, such as moving \todo{samples} between local stations, inserting \todo{dishes} into the platform from a nearby stack or even change header to act as i.e. syringes.\todo{reference}\\
%	
%	Since the 
%	
%	
%	Even though the manipulator, that handles the dishes, are so integrated into the rest of the platform, Since the laboratory platform are very specialized to handle only a certain type of plates, it makes it becomes  mainly are build to handle only one type of dish, it makes it eve
%	Something about not being able to handle different shapes of dishes
%	
%	Laboratory equipment that includes such functionality are often designed to address an industrial standard and will therefore also aim at a high through-put of samples. This severely limits the small users, because the price follows the mentioned functionality and standard.
%	
%	Because the platforms with dish insertion functionality have the deeply integration to the rest of the platform, an institution with a modest budget, or that handles multiple variations of dishes, will not even be able to start by acquiring just some modules and later upgrade with some other modules.
	
	 
	 
	

	
	
	
	
	
	
	
	%The current vessel manipulators are often designed to handle only one specific vessel type - types like Petri dishes or wells (Correct term?). They are also designed towards a high throughput at an industrial level. 
	
	
	
	
%	Buy addressing the mentioned problems, less privileged institutions and institutions with special needs will be able to benefit from the proposed dispenser.
	
	
	\section{Design goals}
	In order to increase the design and clarify the important functionalities needed, a series of user scenarios will be described. The concrete design goals will be extracted from these and kept in mind through-out the design process, and be used to evaluate the proposed system at the end.
	
	\subsection{User scenarios}
	
	\begin{enumerate}
		\item \textbf{Adding dispenser module to a platform}\\
		If a user has a laboratory platform and wants to increase the automation, she can easily integrate the dispenser with only a minor modifications.
		
		\item \textbf{Different types of plates}\\
		To broaden the possible of the dispenser module, it must be able to handle different types of plates. The type of plates that are being handled, must be implicit chosen by the user or completely handled by the module.
		
		\item \textbf{Adding/removing dish stacks}\\
		While a platform and the dispenser is running, it must be straight forward to add and remove stacks of dishes. The user shall not have to be concerned of the state of the dispenser, as the dispenser must adapt to the user's actions in runtime.
		
		\item \textbf{Breakage}\\
		The module may of natural cause be worn and parts may break. When this happens, the parts must be easy and low-lost to replace.
		
	\end{enumerate}
	
	\subsection{Goals}
%	1 - Adding the dispenser after having acquired the lab robot - modular\\
%	2 - Flexible way to add/remove stacks of Petri dished\\
%	Must be easy to access for the user\\
%	Let the user know which stack that may be removed and which are currently being used\\
%	3 - Must address not only one size of Petri dished, but several sized and even wells to support for different experiments \\
%	4 - A part breaks and needs to be replaced\\
	
	%Note: Remember to include software design goals in these!
	%Udgangspunkt i olie-dråbe-eksperimenter  - og andre
	
	\begin{itemize}
		\item \textbf{Modularity} Able to attach/remove the dispenser module without any hassle
		\item \textbf{Versatility }
		\item \textbf{Usability} 	Easy to add/remove stacks of Petri dishes/wells\\
		Detects the platform layers automatically\\
		\item \textbf{Flexibility} Address different sizes of Petri dishes (including wells)
		\item \textbf{Repairability} Easy to create spare parts (low cost)
		
		%Gerne flere kategorier
			
		%Evt simplicitet
		%kontra high performance
	\end{itemize}
	
	
	\section{Mechanical design}
	
	\todo{Why did I choose to use Teensy and not an ATMega with the RAMPS 1.4 => AccelStep-library needs to be called often (argue in this direction)}
	As the proposed module shall be available to the largest amount of people, the design is based on open-source and non-specialized parts\footnote{The open-source parts are available around the world, but can be watched at \url{www.openbuildspartstore.com} and \url{www.pjrc.com/teensy}, which include the blueprints.}. The choice of using open-source parts is too allow the user to either produce them by himself or acquire them elsewhere for a low cost. 
	
	The section will continue in an iterative way, starting by describing the initial thoughts toward the first prototype and the design choices, emphasizing key features like \textit{gripper functionality}, \textit{layer detection} and \textit{dish containers}. At each iteration of the prototype will be evaluated at its key features and on an overall scale.
	
	All of the mechanical design is created using SolidWorks\todo{reference} and can be found at GitHub\todo{Reference} along with the developed software, etc.
	
	\todo{Lægge vægt på at der er brugt "normale" motorer}			
			
	
	\subsection{Initial design}
	\subsubsection{Gripping functionality}	
	The gripping functionality of this proposal is one of the primary functions and a severe amount of prior research has gone into this. Unfortunately, there is not much public available research in this area, for which reason, inspirational thoughts have come from industrial developed and proved designs\todo{references}.
	
	Many of the robots are designed to grip a Petri dish, plate of wells or test tubes by simply squeezing two (often metal) arms from each side of the target.
	
	Those handling well plates\todo{references} have a flat side toward the plate and often have one or two edges/tips at the end. These allow the manipulator to get a hold of, i.e. the bottom of the plate without touching a potential lid, or support the lift with a good hold from the bottom instead of relying on a friction-based grip, that requires additional force added to the sides of the plate. 
	
	The one handling Petri dishes is very similar to the one for well plates, but the "edges" at the end of the arms are a bit concave to have more than two points where the dish is gripped\footnote{If the grip of a Petri dish only has two points, there is a high risk of the dish tipping or even get dropped during the lift and repositioning.}. Some designs have addressed the two-point-issue by adding two vertical plastic cylinders at the bottom of each arm to center the dish while performing the squeeze mechanism. The outer part of the arm, are in some of these cases a bit flexible to help the centering even better, but mainly the end of the arms were fixed.
	
	The manipulators for test tubes reminds of those for Petri dishes, but the form factor was smaller and the arms were all vertically. The end or the arms were also shaped to fit the used test tubes - when closed, the arms would almost be in contact with the test tube all the way around.
	
	One of the end manipulators, which actually \textit{was} from the research world, was PetriJet \todo{reference}. This machine is capable of moving a Petri dish from one stack, to a camera for analysis and on to another stack. The PetriJet uses a suction cup for moving the dishes, which implies that Petri dishes with lids locked were used. The same goes for some of the industrial developed manipulators \todo{reference}, which are used for test tubes, as these were lifted by the rubbery center of the cap. The proposed system should be able to not only handle dishes and plates with lids, but also without lids - therefore a suction cup can't be used.
	
	Among the found liquid handling robots only a single had the capability to change manipulator end at run-time\todo{reference}. The \todo{xxx} was used to not only handle well plates, but also test tubes at the same time, which required the dynamically change of tool for the end manipulator.
	
	\todo{Add figure in this section, showing some of the end manipulators?}
	
	A few of the industrial solutions had designed a small squared plate, where a Petri dish was placed. This prevented the end manipulator from actually grip the dish as a simple fork-inspired design could slide in between two plates, separated by a Petri dish, and lift up the top plate and a the Petri dish placed on top of it. \todo{Picture?}\\
	Such an approach opens op the opportunity to mark each squared plate with a barcode or similar. A picture taken from an experiment could then include this barcode as reference.
	
	To save material and avoid stacks of Petri dishes, that will slide out of position, the squared plate is rejected for now, but noted as a potential solution to forthcoming issues.
		
	Based on the aforementioned background research, the initial design of the gripper will be based on two arms, that respectively will have one and two points of contact with the dish/plate. The end of the arms will be fixed to ensure that, i.e. a misaligned well plate will be correct aligned while the grip is achieved. The reason for having a total of three and not four points of contact is to minimize complexity. Some high-friction material will be added at the contact points to give a better grip.
	
	During the initial design several sensors will be mounted at the tip of the end manipulator. These sensors are of different types, but addresses some of the same issues. Please see \nameref{subsub:Initial_LayerDetection} for details about this.


	\subsubsection{Layer detection}\label{subsub:Initial_LayerDetection}
	Layers, in the proposed system, is to be understood as, i.e. the platform where all experiments are conducted or the layers of tools above/below the platform layer. A layer is static in terms of height but tools and other equipment in the layer can freely be moved around.
	
	Detecting the layers in the proposed module, is to increase the user experience as this otherwise could be done manually and entered in the system of the original laboratory platform. In this way, the proposed module can also be used together with laboratory platforms, where such layers can be adjusted dynamically and complement this by telling the new heights - if the platform itself is not capable of this already.
	
	Amongst the found research projects and industrial robots, none had addressed this issue before. Thoughts go towards, that often the layers are permanently mounted or simply not existing, for which reason it hasn't been seen as a problem. With the constant development of this technology and only more dynamic solutions will be designed, this will still be considered as an important design feature - in this way, the proposed module can also be matched with future laboratory platforms.
	
	As the detection of layers are to be implemented, it might well be used to detect height of dishes/plates in a stack (see \nameref{subsubsec:Initial_DishContainers}). To detect these heights, several solutions will be implemented in the gripper for the initial design. This is to compress the overall process and test multiple solutions at once. The sensors are all standard sensors and of the following types
	
	\begin{itemize}
	\item \textbf{Ultrasound sensors}\footnote{\todo{Add part details}} Ideal for this scenario as physical contact is avoided, the cost is low and precision is high. Two ultrasound (US) are tested, which both has a narrow beam to prevent false distances.
	\item \textbf{IR sensor}\footnote{\todo{Add part details}} This IR will also avoid physical contact, but distances are much closer compared to an US sensor. This particular one has a high-end, build-in noise filter.
	\item \textbf{Level switch}\footnote{\todo{Add part details}} Widely used switch also known as endstop. Requires physical contact, which can be an issue when detecting height of clean dish/plate stacks, but very low cost and reliable.
	\item \textbf{Hall effect sensors}\footnote{\todo{Add part details}} Requires a magnetic element as the target to be detected, which will be an issue when detecting the height of dishes/plates, but low cost, commonly used and no directly physical contact. Two types of hall effect sensors are tested: Switch and linear ratiometric
	\end{itemize}
		
				
	\subsubsection{Dish containers}\label{subsubsec:Initial_DishContainers}
	Make introducing note about it is not one container that will fit it all!
		Room for electronics in the bottom (and protected from liquids from above)
		Angled side + angled cuts to help align dishes that are removed from the platform
		Distance sensors on the top
		
		Starte med modstand der er klistret på i bunden af stakken
			But room for electronics in the bottom!!
			
		Note how the front is symmetrical: During the initial design is was experiences that the plastic material (POM) could be bent. Depending of the severity of this, the front plate can just be flipped, but be careful, that there is room enough for inserting a dish/wells to the stack.
		Nævne skrå front og skæve sider på stak (siderne gør mere end de ser ud til, da de er tættere på centrum en skålens sider)
			
		
		Initial design of the wells container had been designed in a way to minimize the amount of resources. This design *should* be fairly good, but otherwise a design with support in each corner is searched for.
	
	
	\subsubsection{Misc. design decisions}
		Blev løbende klogere under byggeprocessen:
			Mindske huller i POM og bruge POM'en som modpart til skruer
			Fødder
			Lead screws rammer på ramme når de kører hurtigt
	
	
	\subsection{1. iteration}
	asdf
		\subsubsection{Gripping functionality}
			Consider to use a screw with two threadings
			Using too much vertical space:
				change the double alu profile to a single and add corners to prevent it from rotating
		
			Thoughts about two rails for each arm
			
		Arme: Lave yderste huller mindste => vende skruer om og tilføje noget klistret. Så skulle de kunne løfte den inderste skål uden at fjerne toppen først.
		
		\subsubsection{Layer detection}
		adsf
				
		\subsubsection{Dish containers}	
		Considering to rotate the sides by 45 degrees to have a design like
		
		/   \
		\   /
		
			This would ease the gripping and measuring operations as there would be more space. But the use of 2D laser cutting limits the design of "complex" structures.
			
			
		Is the front alu profiel needed? If not (and I think not!!) then the disc containers could get even closer to the rest of the module.
		
		
		Consider to move the alu profiles at the sides into the middle
			This should be duable by defining the distance in the stack-platform
		
		
		Meget smalt område der kan klemmes om når en runde skål skal samles op. Den bør måske gøres bredere (husk at tage højde for stakkens side så)
		
		Ide omkring front: kunne fjernes pga at top forhindre nem adgang til skåle. Vil hellere fjerne top, da slid ellers ville gøre fronten ustabil (kune bruge magneter, men de ville besværliggøre bygningsprocessen)
		
		På stak: forreste alubar kan skydes længere tilbage og gøre plads til en støtte-profil istedet
		
		Moving the alu profile close to each other to save space and allow for more containers in the row.
		
		
		
		Test results:
		
			MB1000: US device needs too long time from transmitting a signal to it can receive a signal. Therefore the US device returned the same distance  from around the top of the sides to the bottom.
			IF (!!) this device is to be used, one could remove the top and simply go higher. This would increase the total high of the module with approximately 30 cm and is therefore rejected
			
			Parallax PING))): Worked very well, but needed to be placed rather precise when measuring from the top of the sides or farther away. 
			Was able to measure down to 1 inch and with a stable output.
			Unfortunately the sensor is big to have mounted on of the gripper arms.
			
			MB1202
			
			VL6180X
			
			Other IR device from sparkfun
			
			Level switch
			
			Hall-effect sensors
			
		Increased distance from top of sides to the actual top of the containter. This is to give more space for the US sensor and because the arms may drop a bit when weight is added.
			
		
		\subsubsection{Misc. design decisions}
		Letting the vertical axis rest on the motor mounts as the lead screw are too steep to hold the vertical axis (plus the rest) by itself.
		The concern about the steep lead screw must be discussed - and addressed!
		
		
		Let the motor module fit several sizes of Nema motors by creating cuttings instead of holes for the screws
		
		Switched the double alu profile for a single alu profile. This increases the stability in the weak, vertical direction and makes the overall size of the "horizontal design" 40mm smaller in the horizontal direction
		
	\subsection{2. iteration}
		asdf
			\subsubsection{Gripping functionality}
			asdf
			
			\subsubsection{Layer detection}
			adsf
					
			\subsubsection{Dish containers}	
			asdf
			
			\subsubsection{Misc. design decisions}
			ASDF
			
	\subsection{Final result}
			Future aspect: Look into wiring and electrical interference!
			\subsubsection{Gripping functionality}
			asdf
			
			\subsubsection{Layer detection}
			adsf
					
			\subsubsection{Dish containers}	
			asdf
			
			\subsubsection{Misc. design decisions}
			ASDF
			
			
	\section{Software design}
	TBD
	
	Using a Teensy LC/Teensy 3.2 because of low cost and can perform at high frequencies
	Going for single threaded for starters.
	
	Aware of the limited space, which is why the system will take in a command and store this, but only in the last second transform a command into several sequential motions (figure?)
	
	
	
	
	Created a queue for commands to be executed
		The commands can then be transferred from the host system to the dispenser to leave the dispenser less independent and the host doesn't have to store upcoming commands.
		This will make it easier to check upcoming motions and see if these can be executed simultaneously

		An OK is returned upon every successfully executed command
		
		
	
	
	Each command line is ended with a semicolon.
	\begin{table}
	\begin{center}
	 \begin{tabular}{|| c | p{5cm} ||} 
	 \hline
	 Command & Description\\ [0.5ex] 
	 \hline\hline
	 d\textless stack number\textgreater  & dispense a dish/plate from stack \textless stack number\textgreater  \\ 
	 \hline
	 r\textless stack number\textgreater & remove dish/plate form platform and put it in stack \textless stack number\textgreater \\
	 \hline
	 c & calibrate \\
	 \hline
	 o & move to origo \\
	 \hline
	 l & get layer positions in mm \\
	 \hline
	 w & get width in mm \\
	 \hline
	 h & get travel height in mm \\ %What we don't know the height of the "bottom" \\
	 \hline
	 p & get current position in mm - format: (\textless width\textgreater ,\textless height\textgreater ) \\
	 \hline
	 s & get amount of dishes in stacks - format: Int array width dish amount for each stack. -1 if no stack in place\\ [1ex] 
	 \hline
	\end{tabular}
	\end{center}
	\caption{Available commands and descriptions.} \label{table:commands}
	\end{table}	
	\todo{Add commands like: goto(x,y,z,g)}

	\section{Testing}
	\subsection{Mechanical precision}
	TBD
	Uddybe hvad en reelt præcision er
	
	
	\subsection{Software testing}
	Different Teensies compared to amount of incoming data?
	
	
	
	\subsection{Integrated experiment}
	TBD
	
	\section{Discussion}
	TBD
	Skal være en mere konceptuel diskussion (relatere til design goals)
	
	
	Make absolutely sure that your supplier can deliver within a few days - otherwise have a small stash yourself!! This can avoid a setback of approximately a month.
	
	\section{Conclusion}
	It is important that you write for the SIGCHI audience.  Please read
	previous years' Proceedings to understand the writing style and
	conventions that successful authors have used.  It is particularly
	important that you state clearly what you have done, not merely what
	you plan to do, and explain how your work is different from previously
	published work, i.e., what is the unique contribution that your work
	makes to the field?  Please consider what the reader will learn from
	your submission, and how they will find your work useful.  If you
	write with these questions in mind, your work is more likely to be
	successful, both in being accepted into the Conference, and in
	influencing the work of our field.
	
	\todo{For electrical schematics contact the author.}
	
	\section{Acknowledgements}
	The author would like to thank primarily Kasper Støy and Andrés Faina for supervising this project, but also PITLab for providing temporarily sensors.
	
	
	% Balancing columns in a ref list is a bit of a pain because you
	% either use a hack like flushend or balance, or manually insert
	% a column break.  http://www.tex.ac.uk/cgi-bin/texfaq2html?label=balance
	% multicols doesn't work because we're already in two-column mode,
	% and flushend isn't awesome, so I choose balance.  See this
	% for more info: http://cs.brown.edu/system/software/latex/doc/balance.pdf
	%
	% Note that in a perfect world balance wants to be in the first
	% column of the last page.
	%
	% If balance doesn't work for you, you can remove that and
	% hard-code a column break into the bbl file right before you
	% submit:
	%
	% http://stackoverflow.com/questions/2149854/how-to-manually-equalize-columns-
	% in-an-ieee-paper-if-using-bibtex
	%
	% Or, just remove \balance and give up on balancing the last page.
	%
	\balance
	
	\bibliographystyle{acm-sigchi}
	\bibliography{references}
\end{document}






